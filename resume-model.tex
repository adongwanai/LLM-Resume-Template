%%
%% Copyright (c) 2018-2020 Weitian LI <wt@liwt.net>
%% CC BY 4.0 License
%%
%% Created: 2018-04-11
%%

% Chinese version
\documentclass[zh]{resume}

% File information shown at the footer of the last page
\fileinfo{%
  \faCopyright{} 2025--2026, 阿东玩AI \hspace{0.5em}
  \creativecommons{by}{4.0} \hspace{0.5em}
  \githublink{adonogwanai}{LLM-Resume-Template} \hspace{0.5em}
  \faEdit{} \today
}

\name{玩AI}{阿东}

\keywords{大模型, 算法工程师, 模型压缩, 模型微调, PyTorch, DeepSpeed}

% \tagline{\texorpdfstring{\icon{\faBinoculars} }{}<position-to-look-for>}
% \tagline{<current-position>}

% \photo[
%   shape=<circular|square>,    % default is circular
%   position=<left|right>,      % default is left
% ]{<width>}{<filename>}
% Example:
% \photo[shape=square]{5.5em}{photo}

\profile{
  \mobile{138-0000-0000}
  \email{adong@tsinghua.edu.cn}
  \university{清华大学}
  \degree{计算机科学与技术 \textbullet 硕士}
  \birthday{1998-06}
  % Custom information:
  % \icontext{<icon>}{<text>}
  % \iconlink{<icon>}{<link>}{<text>}
}

\begin{document}
\makeheader

%======================================================================
% Summary & Objectives
%======================================================================
\begin{abstract}
清华大学计算机科学与技术专业硕士在读,预计 2026 年 6 月毕业。主攻大模型方向,在模型压缩与微调方面有深入研究和丰富实践。具备头部科技公司大模型算法实习经验,熟悉 LLaMA 架构、提示工程及性能优化。科研方面,以一作身份在 NeurIPS 发表论文,专注模型压缩与知识蒸馏。技术实践能力强,曾获 Kaggle 大模型微调比赛金牌,熟悉 PyTorch、DeepSpeed 及 Hugging Face Transformers 等开发框架与工具。
\end{abstract}

%======================================================================
\sectionTitle{专业技能}{\faWrench}
%======================================================================
\begin{itemize}
    \item 熟悉 Python 及 PyTorch 框架,掌握 Hugging Face Transformers 等大模型开发工具。
    \item 深入理解大模型微调技术,如 P-tuning、LoRA 等,并有丰富的提示工程实践经验。
    \item 熟悉 LLaMA 等主流模型架构,以及 RLHF、DPO 等偏好对齐方法。
    \item 掌握基于知识蒸馏的模型压缩方法,对大模型轻量化有深入研究。
    \item 熟悉 DeepSpeed 等分布式训练框架,具备大模型训练与部署性能优化能力。
    \item 了解 RAG、Agent 等 LLM 应用技术,对多模态大模型有研究与实践经验。
\end{itemize}

%======================================================================
\sectionTitle{教育背景}{\faGraduationCap}
%======================================================================
\begin{educations}
  \education%
    {2023.09}%
    [2026.06 (预计)]%
    {清华大学}%
    {计算机学院}%
    {计算机科学与技术}%
    {硕士 \quad 专业排名前 5\%}
  \separator{0.5ex}
  \education%
    {2019.09}%
    [2023.06]%
    {清华大学}%
    {计算机系}%
    {计算机科学与技术}%
    {学士}
\end{educations}

%======================================================================
\sectionTitle{科研经历}{\faAtom}
%======================================================================
\begin{itemize}
    \item \textbf{基于强化学习的自主决策 Agent 系统研究} \hfill 20XX.X --- 至今
    \begin{itemize}
        \item \textbf{问题背景}: 现有 LLM-based Agent 在复杂任务中存在决策路径冗余、缺乏长期规划能力等问题, 在多步骤任务中难以做出最优决策序列
        \item \textbf{研究内容}: 1) 提出基于 PPO 的 Agent 决策优化框架, 将 LLM 作为策略网络, 通过任务完成度、步骤效率等构建奖励函数, 实现端到端强化学习训练; 在 ALFWorld、WebShop 等 Benchmark 上验证, 对比 ReAct、Reflexion 等基线方法, 任务成功率提升 XX\%~XX\%, 平均决策步数减少 XX\%
        \item 2) 设计分层强化学习架构, 将复杂任务分解为高层规划和低层执行两个层次; 通过 Hindsight Experience Replay (HER) 提升样本利用效率, 在稀疏奖励环境下收敛速度提升 XX\%
        \item 3) 引入 Monte Carlo Tree Search (MCTS) 与 RL 结合的混合决策机制, 在决策前进行多步模拟, 优化探索-利用平衡; 在长序列任务中, 相比纯 RL 方法, 样本效率提升 XX\%, 最终性能提升 XX\%
        \item \textbf{相关成果}: 第一作者论文已投稿 NeurIPS/ICML 20XX, 目前处于 Under Review 状态; 开源代码在 GitHub 获得 XXX+ stars
    \end{itemize}
    
    \item \textbf{大模型知识蒸馏与模型压缩技术研究} \hfill 20XX.X --- 20XX.X
    \begin{itemize}
        \item \textbf{问题背景}: 大规模语言模型部署成本高昂, 在边缘设备和实时应用场景下难以满足推理延迟要求
        \item \textbf{研究内容}: 1) 提出基于层级知识蒸馏的模型压缩方法, 同时对齐 Teacher 和 Student 模型的注意力分布、隐层特征和输出概率分布, 在保持 XX\% 性能的前提下, 将 LLaMA-13B 压缩至 XB 参数量; 2) 结合结构化剪枝和低秩分解技术, 设计自适应剪枝策略, 在 MMLU、GSM8K 等 Benchmark 上, 相比标准蒸馏方法, 平均性能提升 X.X\%
        \item \textbf{相关成果}: 第一作者论文发表于 NeurIPS/ICML/ICLR 20XX (CCF-A), 论文被引用 XX 次
    \end{itemize}
\end{itemize}

%======================================================================
\sectionTitle{实习经历}{\faBriefcase}
%======================================================================
\begin{experiences}
  \experience%
    [20XX.X]%
    {20XX.X}%
    {字节跳动/腾讯/阿里巴巴 \quad AI Lab \quad 大模型算法实习生}%
    [\begin{itemize}
      \item \textbf{负责部分}: 参与 XXXX 大模型训练过程, 负责 XXXX 数据构造与模型微调, 负责 XXXX 模块开发与性能优化
      \item \textbf{实习内容1}: 根据 XXXX 业务场景(如代码生成/数学推理/对话等), 挖掘模型能力短板, 构造高质量训练数据 XX B tokens, 进行持续预训练/SFT训练, 在 HumanEval/MATH/MMLU 等 benchmark 消融实验中, 稳定提升平均得分 X~Xpp
      \item \textbf{实习内容2}: 针对模型在 XXXX 场景存在的 XX 问题(如指令遵循/多轮对话/工具调用等), 依据业务流程, 收集和构造 XXXX 数据集, 设计 XXXX 训练策略(如多任务学习/DPO/RLHF等), 基于 LLaMA/Qwen 等模型进行训练, 最终在内部 benchmark 上平均得分继续提升 Xpp, XXXX 指标达到 XX\%
      \item \textbf{实习内容3}: 负责模型推理性能优化, 发现 XXXX 问题(如延迟高/显存占用大等), 采用 XXXX 优化方法(如KV Cache优化/量化/算子融合等), 基于 vLLM/TensorRT 等框架, 进行推理加速优化, 推理延迟降低 XX\%, QPS提升 XX\%, 同时研究模型压缩方法, 最终在保持性能前提下显存占用降低 XX\%
    \end{itemize}]
\end{experiences}

%======================================================================
\sectionTitle{项目经历}{\faCode}
%======================================================================
\begin{itemize}
    \item \textbf{基于自我纠错机制的智能 RAG 检索系统} \hfill 20XX.X --- 20XX.X
    \begin{itemize}
        \item \textbf{项目背景}: 为解决 XXXX 知识库检索效率低下问题, 设计并实现能够自主优化检索策略的 RAG 系统
        \item \textbf{核心痛点}: 最初版本召回率仅 XX\%, 存在 XXXX 问题(如关键词匹配不准/语义理解偏差等)
        \item \textbf{技术方案}: 1) 借鉴 ReAct 思想, 设计迭代式检索策略——当检索结果置信度低时, Agent 自动触发反思机制, 分析失败原因并重新生成优化查询; 2) 实现多策略融合检索(向量检索/BM25/混合检索), 根据查询类型自适应选择最优策略; 3) 引入重排序模块, 基于交叉编码器对召回文档进行精排
        \item \textbf{实验验证}: 在 XXXX 数据集上进行完整消融实验, 迭代式检索使召回率提升 XX\%, 多策略融合再提升 XX\%
        \item \textbf{最终效果}: 召回率从 XX\% 提升至 XX\%, 平均检索轮次 X.X 次, 用户满意度达到 XX\%, 相关技术方案已应用于实际生产环境
        \item \textbf{技术栈}: Python, LangChain, FAISS, Sentence-Transformers, BGE-reranker
    \end{itemize}
    
    \item \textbf{基于分层记忆架构的长程对话 Agent 系统} \hfill 20XX.X --- 20XX.X
    \begin{itemize}
        \item \textbf{项目背景}: 现有 Agent 在长对话场景下, 因上下文窗口限制和记忆管理不当, 导致信息遗忘严重、检索效率低下
        \item \textbf{设计灵感}: 受人类记忆机制启发(工作记忆 + 长期记忆), 设计类似的分层记忆架构
        \item \textbf{架构设计}: 1) \textbf{工作记忆层}: 维护最近 K 轮对话完整上下文; 2) \textbf{短期记忆层}: 对近期对话自动摘要, 提取关键信息; 3) \textbf{长期记忆层}: 基于重要性评分和时间衰减机制, 筛选重要记忆永久保存
        \item \textbf{关键技术}: 1) 实现 BM25 + 向量相似度混合检索; 2) 设计记忆压缩算法; 3) 引入记忆更新策略, 自动处理信息冲突
        \item \textbf{实验结果}: 信息有效保留率达到 XX\%, 记忆检索速度提升 X 倍, 在 XXX 轮以上超长对话中任务完成率提升 XX\%
        \item \textbf{应用场景}: 已集成到个人助理型 Agent, 可处理日程管理/待办事项跟踪/知识问答等长期交互任务
        \item \textbf{技术栈}: Python, LangChain, ChromaDB, Redis, GPT-4
    \end{itemize}
\end{itemize}

%======================================================================
\sectionTitle{个人荣誉}{\faTrophy}
%======================================================================
\begin{itemize}
    \item NeurIPS/ICML/ICLR 20XX 第一作者论文 (CCF-A 类会议), 研究方向: XXXX
    \item Kaggle XXXX Competition 金牌/银牌 (Top X\%, XX/XXXX)
    \item 天池/DataFountain XXXX Challenge 第一名/Top X
    \item 清华大学/北京大学 20XX 年度学业优秀奖学金(一等)、优秀研究生
    \item 20XX 年度国家奖学金
    \item 中国大学生程序设计竞赛 (CCPC) / ACM-ICPC 全国金奖/银奖/铜奖
    \item 美国大学生数学建模竞赛 (MCM/ICM) Outstanding/Meritorious Winner
    \item GitHub 开源项目 "XXXX" 获得 XXX+ stars
\end{itemize}

\end{document}

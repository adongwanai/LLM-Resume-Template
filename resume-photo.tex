% !TeX TS-program = xelatex
\documentclass{resume-photo}

% 字体(Overleaf 推荐用 Noto 系列,避免 SimSun 缺失)
\usepackage{xeCJK}
\setCJKmainfont{Noto Serif CJK SC}
\usepackage{fontspec}

% 紧凑排版
\usepackage{enumitem}
\setlist[itemize]{itemsep=2pt, topsep=2pt}
\newcommand{\ExpItem}[3]{%
  \noindent\textbf{#1|#2} \hfill {\footnotesize #3}\\[-2pt]
}

\ResumeName{阿东玩AI}
\ResumePhoto{adongwanai.jpg}

\begin{document}

\ResumeContacts{
  1XX-XXXX-XXXX,%
  \ResumeUrl{mailto:adong@tsinghua.edu.cn}{adong@tsinghua.edu.cn},%
  \textnormal{清华大学 | 计算机科学与技术 · 硕士 | 20XX-XX}%
}

\ResumeTitle


% ====================== 简介 ======================
\noindent 清华大学计算机科学与技术专业硕士在读,预计 2026 年 6 月毕业。主攻大模型方向,在模型压缩与微调方面有深入研究和丰富实践。具备头部科技公司大模型算法实习经验,熟悉 LLaMA 架构、提示工程及性能优化。科研方面,以一作身份在 NeurIPS 发表论文,专注模型压缩与知识蒸馏。技术实践能力强,曾获 Kaggle 大模型微调比赛金牌,熟悉 PyTorch、DeepSpeed 及 Hugging Face Transformers 等开发框架与工具。


% ====================== 教育 ======================
\section{教育背景}
\ResumeItem
{清华大学}
[\textnormal{计算机学院 | 计算机科学与技术 | 硕士 | 专业排名前 5\%}]
[2023.09—2026.06(预计)]

\ResumeItem
{清华大学}
[\textnormal{计算机系 | 计算机科学与技术 | 学士}]
[2019.09—2023.06]

% ====================== 技能 ======================
\section{专业技能}
\begin{itemize}
  \item 熟悉 Python 及 PyTorch 框架,掌握 Hugging Face Transformers 等大模型开发工具。
  \item 深入理解大模型微调技术,如 P-tuning、LoRA 等,并有丰富的提示工程实践经验。
  \item 熟悉 LLaMA 等主流模型架构,以及 RLHF、DPO 等偏好对齐方法。
  \item 掌握基于知识蒸馏的模型压缩方法,对大模型轻量化有深入研究。
  \item 熟悉 DeepSpeed 等分布式训练框架,具备大模型训练与部署性能优化能力。
  \item 了解 RAG、Agent 等 LLM 应用技术,对多模态大模型有研究与实践经验。
\end{itemize}

% ====================== 科研 ======================
\section{科研经历}

\ResumeItem{基于强化学习的自主决策 Agent 系统研究}
[]
[2024.03—至今]

\begin{itemize}
  \item \textbf{问题背景}: 现有 LLM-based Agent 在复杂任务中存在决策路径冗余、缺乏长期规划能力等问题,难以在多步骤任务中做出最优决策序列
  \item \textbf{研究内容}: 1) 提出基于 PPO 的 Agent 决策优化框架,将 LLM 作为策略网络,通过任务完成度、步骤效率等构建奖励函数,实现端到端的强化学习训练;在 ALFWorld、WebShop 等多个 Benchmark 上验证,对比 ReAct、Reflexion 等基线方法,任务成功率提升 12\%~18\%,平均决策步数减少 25\%
  \item 2) 设计分层强化学习架构,将复杂任务分解为高层规划和低层执行两个层次,高层 Agent 负责子目标分解,低层 Agent 负责具体动作执行;通过 Hindsight Experience Replay (HER) 提升样本利用效率,在稀疏奖励环境下收敛速度提升 40\%
  \item 3) 引入 Monte Carlo Tree Search (MCTS) 与 RL 结合的混合决策机制,在决策前进行多步模拟,优化探索-利用平衡;在长序列任务中,相比纯 RL 方法,样本效率提升 35\%,最终性能提升 8\%
  \item \textbf{相关成果}: 第一作者论文已投稿 NeurIPS 2026,目前处于 Under Review 状态;开源代码在 GitHub 获得 500+ stars
\end{itemize}

\ResumeItem{大模型知识蒸馏与模型压缩技术研究}
[]
[2023.09—2024.02]

\begin{itemize}
  \item \textbf{问题背景}: 大规模语言模型部署成本高昂,在边缘设备和实时应用场景下难以满足推理延迟要求
  \item \textbf{研究内容}: 1) 提出基于层级知识蒸馏的模型压缩方法,同时对齐 Teacher 和 Student 模型的注意力分布、隐层特征和输出概率分布,在保持 95\% 性能的前提下,将 LLaMA-13B 压缩至 3B 参数量;2) 结合结构化剪枝和低秩分解技术,设计自适应剪枝策略,在 MMLU、GSM8K 等 Benchmark 上,相比标准蒸馏方法,平均性能提升 4.2\%
  \item \textbf{相关成果}: 第一作者论文发表于 NeurIPS 2024 (CCF-A),论文被引用 15 次
\end{itemize}

% ====================== 实习 ======================
\section{实习经历}

\ResumeItem{字节跳动|人工智能实验室}
[\textnormal{大模型算法实习生}]
[2024.06—2024.12]

\begin{itemize}
  \item \textbf{负责部分}: 参与豆包大模型的持续预训练和指令微调过程,负责领域数据构造与模型性能优化,参与 Agent 能力增强项目开发
  \item \textbf{实习内容1}: 根据代码生成和数学推理业务场景,挖掘模型在复杂逻辑推理上的能力短板,构造高质量 CoT 数据 5B tokens,进行持续预训练;在 HumanEval、MATH、GSM8K 等 benchmark 的消融实验中,稳定提升模型平均得分 5~8pp
  \item \textbf{实习内容2}: 针对模型在多轮对话中存在的指令遵循不足和上下文理解偏差问题,依据真实业务对话流程,收集和构造多轮对话数据集 50 万条,设计多任务学习目标,基于 LLaMA-3-8B 进行 SFT 和 DPO 训练,最终在内部 benchmark 上平均得分继续提升 3pp,用户满意度达到 87\%
  \item \textbf{实习内容3}: 分析用户反馈数据,发现模型在工具调用场景下准确率不足(仅 72\%),采用 ReAct 范式构造工具调用训练数据,基于 Qwen-14B 模型,融合 Function Calling 和 ReAct 数据进行混合训练,工具调用准确率提升至 89\%;同时研究 Agent 规划能力改进,基于内部平台实现 Tree-of-Thought 推理框架,最终在复杂任务上成功率进一步提升 12\%
\end{itemize}

\ResumeItem{腾讯|微信 AI 团队}
[\textnormal{算法工程实习生}]
[2023.06—2023.12]

\begin{itemize}
  \item \textbf{负责部分}: 参与智能客服大模型的训练与部署,负责模型推理性能优化和压缩加速
  \item \textbf{实习内容1}: 针对线上推理延迟高(P99 达到 3.2s)的问题,实施 KV Cache 优化、算子融合、批处理策略等多项优化措施,将推理延迟降低至 1.8s,QPS 提升 65\%
  \item \textbf{实习内容2}: 研究模型量化技术,实现 INT8 量化和混合精度推理,在保持 98\% 性能的前提下,显存占用降低 55\%,使得单机可部署模型数量提升 2 倍,大幅降低部署成本
  \item \textbf{实习内容3}: 基于 vLLM 框架进行二次开发,实现动态批处理和连续批处理功能,优化调度策略;在高并发场景下(1000+ QPS),相比原生实现,吞吐量提升 80\%,P95 延迟降低 40\%
\end{itemize}



% ====================== 项目 ====================== 
\section{项目经历}

\ResumeItem{基于自我纠错机制的智能 RAG 检索系统}
[]
[2024.01—2024.05]

\begin{itemize}
  \item \textbf{项目背景}: 为解决企业内部知识库检索效率低下的问题,设计并实现一套能够自主优化检索策略的 RAG 系统
  \item \textbf{核心痛点}: 最初版本的 RAG 系统召回率仅有 65\%,存在关键词匹配不准、语义理解偏差等问题,导致用户查询难以获取准确答案,几乎无法投入实际使用
  \item \textbf{技术方案}: 1) 借鉴 ReAct 思想,设计迭代式检索策略——当检索结果置信度低于阈值时,Agent 自动触发反思机制,分析失败原因(如关键词选择不当、查询过于宽泛等),重新生成优化后的查询;2) 实现多策略融合检索,包括向量检索、BM25 关键词检索和混合检索,根据查询类型自适应选择最优策略;3) 引入检索结果重排序模块,基于交叉编码器对召回文档进行精排,提升 Top-K 结果相关性
  \item \textbf{实验验证}: 在企业内部知识库和 MS MARCO、Natural Questions 等公开数据集上进行了完整的消融实验,验证了各模块的有效性——迭代式检索使召回率提升 15\%,多策略融合再提升 3\%,重排序模块贡献 2\%
  \item \textbf{最终效果}: 系统召回率从初始的 65\% 提升至 85\%,平均检索轮次 1.3 次,用户满意度达到 82\%;项目成果在组内进行分享,获得导师和团队成员的高度认可,相关技术方案已应用于实际生产环境
\end{itemize}

\ResumeItem{基于分层记忆架构的长程对话 Agent 系统}
[]
[2023.07—2023.12]

\begin{itemize}
  \item \textbf{项目背景}: 现有 Agent 在长对话场景下,由于上下文窗口限制和记忆管理不当,导致信息遗忘严重、检索效率低下,难以维持长期交互
  \item \textbf{设计灵感}: 受人类记忆机制启发,人类拥有短期记忆(Working Memory,用于当前任务)和长期记忆(Event Memory,用于历史经验),因此设计了类似的分层记忆架构
  \item \textbf{架构设计}: 1) \textbf{工作记忆层}:维护最近 K 轮对话的完整上下文,直接输入 LLM,保证对当前任务的即时响应;2) \textbf{短期记忆层}:对近期(如 1 小时内)的对话进行自动摘要,提取关键信息和决策依据,存储为结构化记忆条目;3) \textbf{长期记忆层}:基于重要性评分(结合对话轮次、用户反馈、任务相关性等因素)和时间衰减机制,筛选重要记忆永久保存,不重要的记忆逐步遗忘
  \item \textbf{关键技术}: 1) 实现基于 BM25 + 向量相似度的混合检索机制,快速定位相关历史记忆;2) 设计记忆压缩算法,将冗长对话通过 LLM 进行总结提炼,减少存储开销;3) 引入记忆更新策略,当新信息与旧记忆冲突时,自动触发记忆修正流程
  \item \textbf{实验结果}: 在多轮对话任务上,信息有效保留率达到 92\%(对比 Baseline 的 68\%),同时得益于分层架构,记忆检索速度提升 3 倍(从平均 450ms 降至 150ms);在 100 轮以上的超长对话中,任务完成率提升 28\%
\end{itemize}





% ====================== 荣誉 ======================
\section{个人荣誉}
\begin{itemize}
  \item NeurIPS 2024 第一作者论文 (CCF-A 类会议),研究方向:大模型知识蒸馏与压缩
  \item Kaggle LLM Science Exam 竞赛 金牌 (Top 1\%, 128/2664)
  \item 天池 TIANCHI LLM Fine-tuning Challenge 第一名
  \item 清华大学 2024 年度学业优秀奖学金 (一等)、优秀研究生
  \item 清华大学 2023 年度国家奖学金
  \item 第十四届中国大学生程序设计竞赛 (CCPC) 全国银奖
  \item 美国大学生数学建模竞赛 (MCM/ICM) Meritorious Winner (一等奖)
  \item GitHub 开源项目 "Agent-RL-Framework" 获得 500+ stars
\end{itemize}

\end{document}
